\section{Apache Atlas}
Apache atlas is the novel adaptable platform which incorporates the center 
set of the functional administration services. The Apache atlas empowers 
the ventures to effectively meet the prerequisites inside the hadoop. 
Additionally, it  delivers the integration of the entire  data environment. 
The database researchers, data analysts, and the data administration group 
can take advantage from the open metadata management and the administration 
capabilities can be utilized for the organizations to create and make the 
catalog of their information resources. These resources can be classified 
and collaborated inside the venture effortlessly
\cite{hid-sp18-412-Apache_Atlas_by_Maven}.

There are three main core components of the Apache Atlas, Type System, 
Graph Engine and Ingest/Export. The type system enables the modelling of 
the metadata for the objects that are intended to be administered. 
The metadata objects are represented by the "entities" which are the 
instances of the "Types". Inside the Apache Atlas the metadata objects are 
managed with the help of the graph model. The rich relationships between the 
metadata objects are taken care by this approach by providing the good 
adaptablity and effective handling of the relationships. Additionally, 
the graph engine also provides the effective indexing by creating the 
relevant indices for the metadata objects with the goal of providing the 
efficient search results. The next component called ingest helps the users 
to post the metadatan to the Atlas. In contrast, the export component will 
help the users to expose the metadata of the Atlas and creates a event 
specific to each change. The end users will able to respond to these 
alterations in the real time by consuming these change events 
\cite{hid-sp18-412-Apache_Atlas_architecture}.
